\documentclass[utf8x]{beamer}

% \usepackage{beamerthemesplit} // Activate for custom appearance
\usepackage[utf8x]{inputenc}
\usepackage[OT1]{fontenc}
\usepackage{graphicx}
\usepackage{listings}
\usepackage{hyperref}
\usepackage{xcolor}

\usetheme{Malmoe}
\usecolortheme{beaver}

\lstloadlanguages{Python,C,sh}

\definecolor{darkgreen}{RGB}{0,93,21}
\definecolor{greenblue}{RGB}{40,110,126}
\definecolor{lightgray}{RGB}{246,246,246}
\definecolor{bordergray}{RGB}{193,193,193}
\definecolor{lightblue}{RGB}{0,114,168}
\definecolor{methblue}{RGB}{0,31,108}


\title{Extending Theano}
\author{Arnaud Bergeron}
\date{\today}

\lstset{
language=Python,
basicstyle=\fontfamily{pcr}\selectfont\footnotesize,
keywordstyle=\color{darkgreen}\bfseries,
commentstyle=\color{greenblue}\itshape,
stringstyle=\color{violet},
showstringspaces=false,
tabsize=4,
backgroundcolor=\color{lightgray},
frame=single,
%emph={theano,MyOp,DoubleOp}, emphstyle=\color{lightblue}\bfseries,
emph={[2]__init__,make_node,perform,infer_shape,c_code,make_thunk,grad,R_op},emphstyle={[2]\color{methblue}},
emph={[3]self},emphstyle={[3]\color{darkgreen}},
moredelim=**[is][{\color{red}}]{`}{`}
}

\newcommand{\code}[1]{\lstinline[emph={[2]}]|#1|}

\begin{document}

\frame[plain]{\titlepage}

\section*{}
\begin{frame}{Outline}
\begin{enumerate}
\item How to Make an Op (Python) (45 min)
\item How to Make an Op (C) (30 min)
\item How to Make a Complex Op (15 min)
\item Optimizations (15 min)
\end{enumerate}
\end{frame}

\section{How to Make an Op (Python)}

\begin{frame}[fragile]{Overview}
\lstinputlisting[lastline=14]{python.py}
\end{frame}

\begin{frame}{\code{__init__}}
\lstinputlisting[firstline=6,lastline=8]{python.py}
\begin{itemize}
\item Optional, a lot of Ops don't have one
\item Serves to set up Op-level parameters
\item Should also perform validation on those parameters
\end{itemize}
\end{frame}

\begin{frame}{\code{__props__}}
\lstinputlisting[firstline=4,lastline=5]{python.py}
\begin{itemize}
\item Optional (although very useful)
\item Generates \code{__hash__}, \code{__eq__} and \code{__str__} methods if present
\item Empty tuple signifies no properties that should take part in comparison
\end{itemize}
\begin{alertblock}{}
Make sure \code{__hash__}, \code{__eq__} and \code{__str__} are not defined in a superclass if you don't inherit directly from Op since otherwise your methods will get shadowed.
\end{alertblock}
\end{frame}

\begin{frame}{\code{make_node}}
\lstinputlisting[firstline=9,lastline=11]{python.py}
\begin{itemize}
\item This creates the node object that represents our computation in the graph
\item The parameters are usually Theano variables, but can be python objects too
\item The return value must be an \code{Apply} instance
\end{itemize}
\end{frame}

\begin{frame}{\code{perform}}
\lstinputlisting[firstline=12,lastline=14]{python.py}
\begin{itemize}
\item This performs the computation on a set of values (hence the method name)
\item The parameters are all python objects (not symbolic values)
\item This method must not return its result, but rather store it in the 1-element lists (or cells) provided in \code{output_storage}
\end{itemize}
\end{frame}

\begin{frame}{DoubleOp}
\lstinputlisting[lastline=15]{doubleop.py}
\end{frame}

\begin{frame}{Op Instances and Nodes}
When you call an op class you get an instance of that Op:
\vskip4mm
\hskip3em\code{double_op = DoubleOp()}
\vskip4mm
But when you want to use that op as a node in a graph you need to call the \textit{instance}:
\vskip4mm
\hskip3em\code{node = double_op(x)}
\vskip4mm
You can do both steps at once with a double call like this:
\vskip4mm
\hskip3em\code{node = DoubleOp()(x)}
\end{frame}

\begin{frame}{Basic Tests}
\lstinputlisting[linerange={1-4,7-17}]{test_doubleop.py}
\end{frame}

\begin{frame}[fragile]{Run Tests}
The simplest way to run your tests is to use \texttt{nosetests} directly on your test file like this:

\begin{lstlisting}[language={},backgroundcolor=\color{white},frame={}]
$ nosetests test_doubleop.py
.
------------------------------------------------------
Ran 1 test in 0.427s

OK
\end{lstlisting}

You can also use \texttt{theano-nose} which is a wrapper around \texttt{nosetests} with some extra options.

\end{frame}

\begin{frame}{Exercise: TripleOp}
What would need to be changed in the code below (DoubleOp) to make this Op triple the input instead of double?
\lstinputlisting[lastline=15]{doubleop.py}
\end{frame}

\begin{frame}{Solution: TripleOp}
You change the class name and the constant \code{2} for a constant \code{3}. \\
\ 
\lstinputlisting[lastline=15]{tripleop.py}
\end{frame}

\begin{frame}{Exercise: ScalMulOp}
Now what would we need to change in the DoubleOp code to make it multiply by an arbitrary number?
\lstinputlisting[lastline=15]{doubleop.py}
\end{frame}

\begin{frame}{\code{infer_shape}}
\lstinputlisting[firstline=15,lastline=17]{python.py}
\begin{itemize}
\item This functions is optional, although highly recommended
\item It takes as input the symbolic shapes of the input variables
\item \code{input_shapes} is of the form \code{[[i0_shp0, i0_shp1, ...], ...]}
\item It must return a list with the symbolic shape of the output variables
\end{itemize}
\end{frame}

\begin{frame}{Example}
\lstinputlisting[firstline=16,lastline=18]{doubleop.py}
\begin{itemize}
\item Here the code is really simple since we don't change the shape in any way in our Op
\item \code{input_shapes} would be \code{[x.shape]} or an expression equivalent to it
\end{itemize}
\end{frame}

\begin{frame}{Tests}
To test the \code{infer_shape} method we use \code{InferShapeTester}
\lstinputlisting[linerange={5-5,18-31}]{test_doubleop.py}
\end{frame}

\begin{frame}{Gradient}
\lstinputlisting[firstline=18,lastline=20]{python.py}
\begin{itemize}
\item This function is required for graphs including your op to work with \code{theano.grad()}
\item It must return a list of symbolic graphs for each of your inputs
\item You compute the partial derivative with regards to your inputs
\item Inputs that have no valid gradient should have a special \code{DisconnectedType} value
\end{itemize}
\end{frame}

\begin{frame}{Example}
\lstinputlisting[firstline=19,lastline=21]{doubleop.py}
\begin{itemize}
\item Here since the operation is simple the gradient is simple
\item Note that we return a list
\end{itemize}
\end{frame}

\begin{frame}{Tests}
To test the gradient we use \code{verify_grad}
\lstinputlisting[linerange={5-5,33-41}]{test_doubleop.py}
It will compute the gradient numerically and symbolically (using our \code{grad()} method) and compare the two.
\end{frame}

%\subsection{\code{R_op}}

%\begin{frame}{Description}
%\end{frame}

%\begin{frame}{Example}
%\end{frame}

%\begin{frame}{Tests}
%\end{frame}

\begin{frame}{Add Special Methods to ScalMulOp}
\begin{itemize}
\item Take the ScalMulOp class you made and add the \code{infer_shape} and \code{grad} methods to it.
\item Don't forget to make tests for your new class to make sure everything works correctly.
\end{itemize}
\end{frame}

\section{How to Make an Op (C)}

\begin{frame}{Overview}
\lstinputlisting{c.py}
\end{frame}

\begin{frame}{\code{c_code}}
\lstinputlisting[linerange={9-11}]{c.py}
\begin{itemize}
\item This method returns a python string containing C code
\item \code{input_names} contains the variable names where the inputs are
\item \code{output_names} contains the variable names where to place the outputs
\item \code{sub} contains some code snippets to insert into our code (mostly to indicate failure)
\end{itemize}
\end{frame}

\begin{frame}{Support Code}
\lstinputlisting[linerange={13-14}]{c.py}
\begin{itemize}
\item This method return a python string containing C code
\item The code may be shared with multiple instances of the op
\item It can contain things like helper functions
\end{itemize}
There are a number of similar methods to insert code at various points
\end{frame}

\begin{frame}{Headers, Libraries, Compilers}
Some of the methods available to customize the compilation environment:
\begin{description}
\item[\texttt{c\_libraries}] Return a list of shared libraries the op needs
\item[\texttt{c\_headers}] Return a list of included headers the op needs
\item[\texttt{c\_compiler}] C compiler to use (if not the default)
\end{description}
Again others are available.  Refer to the documentation for a complete list.
\end{frame}

\begin{frame}[allowframebreaks]{Example}
\vskip5mm
This is the C code equivalent to \code{perform}
\vskip4mm
\lstinputlisting[linerange={1-27}]{doublec.py}
\end{frame}

\begin{frame}{Exercice: DoubleC}
\begin{itemize}
\item Make a new DoubleC op that only accepts vectors as input using the C code above.
\item Copy and modify the tests for DoubleC.  Be sure to check for invalid inputs (matrices).
\end{itemize}
\end{frame}

\begin{frame}{COp}
\lstinputlisting{cop.py}
\end{frame}

\begin{frame}{Constructor Arguments}
\begin{itemize}
\item Basically you just pass two arguments to the constructor of COp
\begin{itemize}
\item Either by calling the constructor directly \code{COp.__init__(self, ...)}
\item Or via the superclass \code{super(MyOp, self).__init__(...)}
\end{itemize}
\item The two arguments are:
\begin{itemize}
\item the name of the C code file
\item the name of the function to call to make the computation
\end{itemize}
\end{itemize}
\end{frame}

\begin{frame}{COp: Example}
\only<1>{\lstinputlisting[linerange={1-16}]{doublecop.py}}
\only<2>{\lstinputlisting[language=C]{doublecop.c}}
\end{frame}

\begin{frame}{Tests}
\begin{itemize}
\item Testing ops with C code is done the same way as testing for python ops
\item One thing to watch for is that tests for ops which don't have python code
\begin{itemize}
\item You should skip the test in those cases
\end{itemize}
\item Using DebugMode will compare the output of the Python version to the output of the C version and raise an error if they don't match
\end{itemize}
\end{frame}

\begin{frame}{Gradient and Other Concerns}
\begin{itemize}
\item The code for \code{grad()} and \code{infer_shape()} is done the same way as for a python Op
\item In fact you can have the same Op with a python and a C version sharing the \code{grad()} and \code{infer_shape()} code
\begin{itemize}
\item That's how most Ops are implemented
\end{itemize}
\end{itemize}
\end{frame}

\begin{frame}{Add C Code to ScalMulOp}
\begin{itemize}
\item Take the ScalMulOp from before and write C code for it using either approach
\item You can base yourself on the C code for DoubleOp
\item Don't forget to test your new implementation
\end{itemize}
\end{frame}

\section{How to make a complex Op}

\begin{frame}{\code{make_thunk}}
\lstinputlisting[linerange={12-14}]{thunk.py}
\begin{itemize}
\item Define instead of \code{perform} or \code{c_code}
\item Gives total freedom on how the computation is performed
\item More complex to use and generally not needed
\end{itemize}
\end{frame}

%\begin{frame}{Example}
%\color{red}Once again with ScalMulOp but with a make thunk that passes the scalar value to a helper function
%\end{frame}

\section{Optimizations}
% How to integrate your op automatically

\begin{frame}{Replace an Existing Op}
% use theano.gof.opt.OpSub
\color{red} Replace ScalMulOp(2) with DoubleOp
\end{frame}

\end{document}
